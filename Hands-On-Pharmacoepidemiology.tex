% Options for packages loaded elsewhere
\PassOptionsToPackage{unicode}{hyperref}
\PassOptionsToPackage{hyphens}{url}
\PassOptionsToPackage{dvipsnames,svgnames,x11names}{xcolor}
%
\documentclass[
  14pt,
  letterpaper,
  DIV=11,
  numbers=noendperiod]{scrreprt}

\usepackage{amsmath,amssymb}
\usepackage{iftex}
\ifPDFTeX
  \usepackage[T1]{fontenc}
  \usepackage[utf8]{inputenc}
  \usepackage{textcomp} % provide euro and other symbols
\else % if luatex or xetex
  \usepackage{unicode-math}
  \defaultfontfeatures{Scale=MatchLowercase}
  \defaultfontfeatures[\rmfamily]{Ligatures=TeX,Scale=1}
\fi
\usepackage{lmodern}
\ifPDFTeX\else  
    % xetex/luatex font selection
\fi
% Use upquote if available, for straight quotes in verbatim environments
\IfFileExists{upquote.sty}{\usepackage{upquote}}{}
\IfFileExists{microtype.sty}{% use microtype if available
  \usepackage[]{microtype}
  \UseMicrotypeSet[protrusion]{basicmath} % disable protrusion for tt fonts
}{}
\makeatletter
\@ifundefined{KOMAClassName}{% if non-KOMA class
  \IfFileExists{parskip.sty}{%
    \usepackage{parskip}
  }{% else
    \setlength{\parindent}{0pt}
    \setlength{\parskip}{6pt plus 2pt minus 1pt}}
}{% if KOMA class
  \KOMAoptions{parskip=half}}
\makeatother
\usepackage{xcolor}
\setlength{\emergencystretch}{3em} % prevent overfull lines
\setcounter{secnumdepth}{5}
% Make \paragraph and \subparagraph free-standing
\makeatletter
\ifx\paragraph\undefined\else
  \let\oldparagraph\paragraph
  \renewcommand{\paragraph}{
    \@ifstar
      \xxxParagraphStar
      \xxxParagraphNoStar
  }
  \newcommand{\xxxParagraphStar}[1]{\oldparagraph*{#1}\mbox{}}
  \newcommand{\xxxParagraphNoStar}[1]{\oldparagraph{#1}\mbox{}}
\fi
\ifx\subparagraph\undefined\else
  \let\oldsubparagraph\subparagraph
  \renewcommand{\subparagraph}{
    \@ifstar
      \xxxSubParagraphStar
      \xxxSubParagraphNoStar
  }
  \newcommand{\xxxSubParagraphStar}[1]{\oldsubparagraph*{#1}\mbox{}}
  \newcommand{\xxxSubParagraphNoStar}[1]{\oldsubparagraph{#1}\mbox{}}
\fi
\makeatother

\usepackage{color}
\usepackage{fancyvrb}
\newcommand{\VerbBar}{|}
\newcommand{\VERB}{\Verb[commandchars=\\\{\}]}
\DefineVerbatimEnvironment{Highlighting}{Verbatim}{commandchars=\\\{\}}
% Add ',fontsize=\small' for more characters per line
\usepackage{framed}
\definecolor{shadecolor}{RGB}{241,243,245}
\newenvironment{Shaded}{\begin{snugshade}}{\end{snugshade}}
\newcommand{\AlertTok}[1]{\textcolor[rgb]{0.68,0.00,0.00}{#1}}
\newcommand{\AnnotationTok}[1]{\textcolor[rgb]{0.37,0.37,0.37}{#1}}
\newcommand{\AttributeTok}[1]{\textcolor[rgb]{0.40,0.45,0.13}{#1}}
\newcommand{\BaseNTok}[1]{\textcolor[rgb]{0.68,0.00,0.00}{#1}}
\newcommand{\BuiltInTok}[1]{\textcolor[rgb]{0.00,0.23,0.31}{#1}}
\newcommand{\CharTok}[1]{\textcolor[rgb]{0.13,0.47,0.30}{#1}}
\newcommand{\CommentTok}[1]{\textcolor[rgb]{0.37,0.37,0.37}{#1}}
\newcommand{\CommentVarTok}[1]{\textcolor[rgb]{0.37,0.37,0.37}{\textit{#1}}}
\newcommand{\ConstantTok}[1]{\textcolor[rgb]{0.56,0.35,0.01}{#1}}
\newcommand{\ControlFlowTok}[1]{\textcolor[rgb]{0.00,0.23,0.31}{\textbf{#1}}}
\newcommand{\DataTypeTok}[1]{\textcolor[rgb]{0.68,0.00,0.00}{#1}}
\newcommand{\DecValTok}[1]{\textcolor[rgb]{0.68,0.00,0.00}{#1}}
\newcommand{\DocumentationTok}[1]{\textcolor[rgb]{0.37,0.37,0.37}{\textit{#1}}}
\newcommand{\ErrorTok}[1]{\textcolor[rgb]{0.68,0.00,0.00}{#1}}
\newcommand{\ExtensionTok}[1]{\textcolor[rgb]{0.00,0.23,0.31}{#1}}
\newcommand{\FloatTok}[1]{\textcolor[rgb]{0.68,0.00,0.00}{#1}}
\newcommand{\FunctionTok}[1]{\textcolor[rgb]{0.28,0.35,0.67}{#1}}
\newcommand{\ImportTok}[1]{\textcolor[rgb]{0.00,0.46,0.62}{#1}}
\newcommand{\InformationTok}[1]{\textcolor[rgb]{0.37,0.37,0.37}{#1}}
\newcommand{\KeywordTok}[1]{\textcolor[rgb]{0.00,0.23,0.31}{\textbf{#1}}}
\newcommand{\NormalTok}[1]{\textcolor[rgb]{0.00,0.23,0.31}{#1}}
\newcommand{\OperatorTok}[1]{\textcolor[rgb]{0.37,0.37,0.37}{#1}}
\newcommand{\OtherTok}[1]{\textcolor[rgb]{0.00,0.23,0.31}{#1}}
\newcommand{\PreprocessorTok}[1]{\textcolor[rgb]{0.68,0.00,0.00}{#1}}
\newcommand{\RegionMarkerTok}[1]{\textcolor[rgb]{0.00,0.23,0.31}{#1}}
\newcommand{\SpecialCharTok}[1]{\textcolor[rgb]{0.37,0.37,0.37}{#1}}
\newcommand{\SpecialStringTok}[1]{\textcolor[rgb]{0.13,0.47,0.30}{#1}}
\newcommand{\StringTok}[1]{\textcolor[rgb]{0.13,0.47,0.30}{#1}}
\newcommand{\VariableTok}[1]{\textcolor[rgb]{0.07,0.07,0.07}{#1}}
\newcommand{\VerbatimStringTok}[1]{\textcolor[rgb]{0.13,0.47,0.30}{#1}}
\newcommand{\WarningTok}[1]{\textcolor[rgb]{0.37,0.37,0.37}{\textit{#1}}}

\providecommand{\tightlist}{%
  \setlength{\itemsep}{0pt}\setlength{\parskip}{0pt}}\usepackage{longtable,booktabs,array}
\usepackage{calc} % for calculating minipage widths
% Correct order of tables after \paragraph or \subparagraph
\usepackage{etoolbox}
\makeatletter
\patchcmd\longtable{\par}{\if@noskipsec\mbox{}\fi\par}{}{}
\makeatother
% Allow footnotes in longtable head/foot
\IfFileExists{footnotehyper.sty}{\usepackage{footnotehyper}}{\usepackage{footnote}}
\makesavenoteenv{longtable}
\usepackage{graphicx}
\makeatletter
\def\maxwidth{\ifdim\Gin@nat@width>\linewidth\linewidth\else\Gin@nat@width\fi}
\def\maxheight{\ifdim\Gin@nat@height>\textheight\textheight\else\Gin@nat@height\fi}
\makeatother
% Scale images if necessary, so that they will not overflow the page
% margins by default, and it is still possible to overwrite the defaults
% using explicit options in \includegraphics[width, height, ...]{}
\setkeys{Gin}{width=\maxwidth,height=\maxheight,keepaspectratio}
% Set default figure placement to htbp
\makeatletter
\def\fps@figure{htbp}
\makeatother

\usepackage{fvextra}
\DefineVerbatimEnvironment{Highlighting}{Verbatim}{breaklines,commandchars=\\\{\}}
\areaset[0.50in]{4.5in}{8in}
\KOMAoption{captions}{tableheading}
\makeatletter
\@ifpackageloaded{bookmark}{}{\usepackage{bookmark}}
\makeatother
\makeatletter
\@ifpackageloaded{caption}{}{\usepackage{caption}}
\AtBeginDocument{%
\ifdefined\contentsname
  \renewcommand*\contentsname{Table of contents}
\else
  \newcommand\contentsname{Table of contents}
\fi
\ifdefined\listfigurename
  \renewcommand*\listfigurename{List of Figures}
\else
  \newcommand\listfigurename{List of Figures}
\fi
\ifdefined\listtablename
  \renewcommand*\listtablename{List of Tables}
\else
  \newcommand\listtablename{List of Tables}
\fi
\ifdefined\figurename
  \renewcommand*\figurename{Figure}
\else
  \newcommand\figurename{Figure}
\fi
\ifdefined\tablename
  \renewcommand*\tablename{Table}
\else
  \newcommand\tablename{Table}
\fi
}
\@ifpackageloaded{float}{}{\usepackage{float}}
\floatstyle{ruled}
\@ifundefined{c@chapter}{\newfloat{codelisting}{h}{lop}}{\newfloat{codelisting}{h}{lop}[chapter]}
\floatname{codelisting}{Listing}
\newcommand*\listoflistings{\listof{codelisting}{List of Listings}}
\makeatother
\makeatletter
\makeatother
\makeatletter
\@ifpackageloaded{caption}{}{\usepackage{caption}}
\@ifpackageloaded{subcaption}{}{\usepackage{subcaption}}
\makeatother

\ifLuaTeX
  \usepackage{selnolig}  % disable illegal ligatures
\fi
\usepackage{bookmark}

\IfFileExists{xurl.sty}{\usepackage{xurl}}{} % add URL line breaks if available
\urlstyle{same} % disable monospaced font for URLs
\hypersetup{
  pdftitle={Hands-On Pharmacoepidemiology},
  pdfauthor={Ignacio Leiva},
  colorlinks=true,
  linkcolor={blue},
  filecolor={Maroon},
  citecolor={Blue},
  urlcolor={Blue},
  pdfcreator={LaTeX via pandoc}}


\title{Hands-On Pharmacoepidemiology}
\usepackage{etoolbox}
\makeatletter
\providecommand{\subtitle}[1]{% add subtitle to \maketitle
  \apptocmd{\@title}{\par {\large #1 \par}}{}{}
}
\makeatother
\subtitle{Practical Exercises and Case Studies}
\author{Ignacio Leiva}
\date{2023-10-19}

\begin{document}
\maketitle

\RecustomVerbatimEnvironment{verbatim}{Verbatim}{
   showspaces = false,
   showtabs = false,
   breaksymbolleft={},
   breaklines
   % Note: setting commandchars=\\\{\} here will cause an error 
}  

\renewcommand*\contentsname{Table of contents}
{
\hypersetup{linkcolor=}
\setcounter{tocdepth}{2}
\tableofcontents
}

\bookmarksetup{startatroot}

\chapter*{Practical exercises}\label{practical-exercises}
\addcontentsline{toc}{chapter}{Practical exercises}

\markboth{Practical exercises}{Practical exercises}

This hands-on workbook has been designed to help you put key PhEpi
concepts and methods into practice. This workbook supports the material
taught in the pharmacoepidemiology course and provides a practical
application framework for understanding the topics covered in lectures.
The exercises bridge the gap between theory and practice, offering
guided opportunities for deepening comprehension.

\section*{Lectures}\label{lectures}
\addcontentsline{toc}{section}{Lectures}

\markright{Lectures}

\href{https://ignaleiva.github.io/Lecture2-Frequency-Effect/\#/section}{Lecture
2. Measures in (Pharmaco)Epidemiology}

\bookmarksetup{startatroot}

\chapter{Measures of frequency}\label{sec-measure-of-frequency}

\subsection{Incidence}\label{incidence}

\begin{Shaded}
\begin{Highlighting}[]
\DecValTok{1}\SpecialCharTok{+}\DecValTok{2}
\end{Highlighting}
\end{Shaded}

\begin{verbatim}
[1] 3
\end{verbatim}

\subsection{Prevalence}\label{prevalence}

\bookmarksetup{startatroot}

\chapter{Measures of effect (association)}\label{sec-measure-of-effect}

\section{Risk Ratios}\label{risk-ratios}

\subsection{Excercise 1}\label{excercise-1}

You are part of a research team conducting a cohort study to evaluate
the effect of hormone replacement therapy (HRT) on the development of
coronary heart disease (CHD). The research team enrolled post-menopausal
women on HRT with no prior history of CHD and followed them for 7 years.

Sssume no loss to follow-up.

Remember: Exposure, HRT, is on the left side of the table, while the
outcome, CHD, is on the top of the table.

\begin{longtable}[]{@{}llll@{}}
\toprule\noalign{}
\endhead
\bottomrule\noalign{}
\endlastfoot
& CHD (+) & CHD (-) & Total \\
HRT (+) & 210 & 3290 & 3500 \\
HRT(-) & 250 & 6250 & 6500 \\
Total & 460 & 9540 & 10000 \\
\end{longtable}

\emph{Task:}

Calculate and interpret the relative risk (RR) for the cohort.

Results

\[
P(D+ \mid E+)= \frac{210}{3500} = 0.060
\]

\[
P(D+ \mid E-)= \frac{250}{6500} = 0.038
\]

\[
RR= \frac{0.64}{0.4} = 1.58
\]

\textbf{Interpretation:} Women on HRT have 1.58 times the risk of CHD
compared to those who do not take HRT over 7 years.

Calculate and interpret the risk difference (RD) for the cohort

Results

\[
RD= 0.060 - 0.038 = 0.022
\]

\textbf{Interpretation:} There is an excess of 22 cases of CHD per 1000
women attributable to HRT use over 7 years.

\subsection{Excercise 2}\label{excercise-2}

A study assesses the relationship between smoking habits and esophageal
cancer. 1200 people were enrolled in the study. 450 out of 1200
participants had esophageal cancer, while 320 of the diseased
participants were smokers while. The total number of smokers in the
study was 550.

Fill out the 2x2 table!

\begin{longtable}[]{@{}llll@{}}
\toprule\noalign{}
\endhead
\bottomrule\noalign{}
\endlastfoot
& Lung Cancer (+) & Lung Cancer (-) & Total \\
Smoker (+) & & & \\
Smoker(-) & & & \\
Total & & & \\
\end{longtable}

Results

\begin{longtable}[]{@{}llll@{}}
\toprule\noalign{}
\endhead
\bottomrule\noalign{}
\endlastfoot
& Lung Cancer (+) & Lung Cancer (-) & Total \\
Smoker (+) & 320 & 230 & 550 \\
Smoker(-) & 130 & 520 & 650 \\
Total & 450 & 750 & 1200 \\
\end{longtable}

\emph{Task:}

What proportion of the incidence of esophageal cancer can be directly
attributed to smoking?

Results

\[
\text{Atributable Risk} = \frac{320}{550} - \frac{130}{650} =  0.58 – 0.20 = 0.38
\]

\textbf{Interpretation:} 0.38 or 38\% of the cases of esophageal cancer
in the cohort can be attributed to smoking

Calculate the risk ratio for the relationship between smoking and lung
cancer, interpret the result.

Results

\[
RR = \frac{\frac{320}{550}}{\frac{130}{650}} = \frac{0.58}{0.20} = 2.90
\]

\textbf{Interpretation:} The risk of esophageal cancer among the smokers
is 2,90 times as high as the risk of esophageal cancer among the
non-smokers.

\subsection{Excercise 3}\label{excercise-3}

Coming back the investigation of the effect of HRT in post-menopausal
women, you now have the following information

\begin{longtable}[]{@{}lll@{}}
\toprule\noalign{}
\endhead
\bottomrule\noalign{}
\endlastfoot
& With CHD (+) & Person-Years of Disease-free Follow-up \\
HRT (+) & 28 & 52,106 \\
HRT (-) & 58 & 50,238 \\
\end{longtable}

\emph{Task:}

Calculate the Incidece rate among the exposed and unexposed women

Results

\[
IR_{exposed}= \frac{28}{52,106} = 0.0005373661 \times 100,000 =  53.74 \text{ person-years}
\]

\[
IR_{unexposed}= \frac{58}{50,238} = 0.001154505 \times 100,000 = 115.45 \text{ person-years}
\]

Calculate the rate ratio for the relationship between HRT and CHD,
interpret the result.

Results

\[
\text{Rate Ratio}= \frac{53.74}{115.45}  = 0.465
\]

\textbf{Interpretation:} Women on HRT had 0.47 times the rate of CHD
compared to women who did not use HRT

\section{Odds Ratios}\label{odds-ratios}

\subsection{Excercise 1}\label{excercise-1-1}

The influence of increased alcohol consumption on the incidence of
esophageal cancer was retrospectively investigated in a clinic over a
period of 5 years. Patients with other internal diseases were used as a
comparison group.

\begin{longtable}[]{@{}llll@{}}
\toprule\noalign{}
\endhead
\bottomrule\noalign{}
\endlastfoot
& Cancer (+) & Cancer (-) & Total \\
Alcohol (+) & 192 & 54 & 246 \\
Alcohol(-) & 208 & 333 & 541 \\
Total & 400 & 387 & 787 \\
\end{longtable}

\emph{Task:}

Calculate the odds to be exposed while diseased/not diseased
(i.e.~case/control)

Results

\[
Odds_{case} = \frac{192}{208} = 0.92
\]

\[
Odds_{control} = \frac{54}{333} = 0.16
\]

Calculate and interpret the odds ratio for the occurrence of esophageal
cancer

Results

\[
OR = \frac{0.92}{0.16} = 5.75
\]

\textbf{Interpretation:}

\begin{itemize}
\item
  The odds of being exposed to increased alcohol consumption among cases
  (patients with cancer) is 5.75 times as high compared to the controls
  (patients with other internal diseases).
\item
  The odds of having cancer are 5.75 times higher among individuals with
  increased alcohol consumption compared to those without increased
  alcohol consumption.
\end{itemize}

\subsection{Excercise 2}\label{excercise-2-1}

A cohort study is investigating the impact of regular physical activity
on the incidence of heart disease A total of 1500 individuals were
enrolled. The prevalence of heart disease was 25\%. 560 of the total
participants were in the active group (those practicing physical
activity). Among that group, 200 developed heart disease.

Fill out the 2x2 table!

\begin{longtable}[]{@{}llll@{}}
\toprule\noalign{}
\endhead
\bottomrule\noalign{}
\endlastfoot
& Phy. Activity (+) & Phy. Activity (-) & Total \\
Smoker (+) & & & \\
Smoker(-) & & & \\
Total & & & \\
\end{longtable}

Results

\begin{longtable}[]{@{}llll@{}}
\toprule\noalign{}
\endhead
\bottomrule\noalign{}
\endlastfoot
& Phy. Activity (+) & Phy. Activity (-) & Total \\
Smoker (+) & 200 & 360 & 560 \\
Smoker(-) & 175 & 765 & 940 \\
Total & 560 & 940 & 1500 \\
\end{longtable}

\emph{Task:}

Calculate the risk for each group and the relative risk based on the
table you filled out, interpret the results.

Results

\[
Risk_{exposed} = \frac{200}{560} = 0.36
\]

\[
Risk_{unexposed} = \frac{175}{940} = 0.19
\]

\[
RR = \frac{0.33}{0.12} = 1.89
\]

\textbf{Interpretation:} The risk of suffering from heart disease when
practicing physical activity is 1.89 times as high compared to
participants who didn't practice physical activity

Moving to the case-control setting, assume a case-control study instead
of a cohort study. Using the same previous figures. Calculate the odds
and odds ratio. Please interpret your results.

Results

\[
Odds_{case} = \frac{200}{175} = 1.14
\]

\[
Odds_{control} = \frac{360}{765} = 0.47
\]

\[
OR = \frac{1.67}{0.45} = 2.42
\]

\textbf{Interpretation:} The odds of practicing physical activity among
cases is 2.42 times as high as the odd of practicing physical activity
among the controls.

\subsection{Excercise 3}\label{excercise-3-1}

A second study is again investigating the influence of smoking on
respiratory complaints. This time, 900 study participants were enrolled,
93\% of those didn't suffer from respiratory complaints. 5\% of the 600
participants that were smokers suffered from respiratory complaints.

Fill out the 2x2 table!

\begin{longtable}[]{@{}llll@{}}
\toprule\noalign{}
\endhead
\bottomrule\noalign{}
\endlastfoot
& Resp. Complaints (+) & Resp. Complaints (-) & Total \\
Smoker (+) & & & \\
Smoker(-) & & & \\
Total & & & \\
\end{longtable}

Results

\begin{longtable}[]{@{}llll@{}}
\toprule\noalign{}
\endhead
\bottomrule\noalign{}
\endlastfoot
& Resp. Complaints (+) & Resp. Complaints (-) & Total \\
Smoker (+) & 30 & 570 & 600 \\
Smoker(-) & 33 & 267 & 300 \\
Total & 63 & 837 & 900 \\
\end{longtable}

\emph{Task:}

Calculate the risk ratio and the odds ratio

Results

\[
RR = \frac{\frac{30}{600}}{\frac{33}{300}} = 0.45
\]

\[
OR = \frac{\frac{30}{570}}{\frac{33}{267}} = 0.43
\]

Looking at the different measures you calculated, explain why the RR and
OR are different/similar from each other in task 2 vs.~task 3.

Results

\textbf{Rare disease assumption:} when studying diseases that have a low
prevalence (\textless10\% as a rule of thumb) in the study population,
the odds ratio is a good approximation of the risk ratio.

\bookmarksetup{startatroot}

\chapter{Bias \& Condounding}\label{sec-bias}

\subsection{Incidence}\label{incidence-1}

\subsection{Prevalence}\label{prevalence-1}




\end{document}
