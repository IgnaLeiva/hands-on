% Options for packages loaded elsewhere
\PassOptionsToPackage{unicode}{hyperref}
\PassOptionsToPackage{hyphens}{url}
%
\documentclass[
  letterpaper,
  paper=6in:9in,
  pagesize=pdftex,
  headinclude=on,
  footinclude=on,
  12pt]{scrbook}

\usepackage{amsmath,amssymb}
\usepackage{iftex}
\ifPDFTeX
  \usepackage[T1]{fontenc}
  \usepackage[utf8]{inputenc}
  \usepackage{textcomp} % provide euro and other symbols
\else % if luatex or xetex
  \usepackage{unicode-math}
  \defaultfontfeatures{Scale=MatchLowercase}
  \defaultfontfeatures[\rmfamily]{Ligatures=TeX,Scale=1}
\fi
\usepackage{lmodern}
\ifPDFTeX\else  
    % xetex/luatex font selection
\fi
% Use upquote if available, for straight quotes in verbatim environments
\IfFileExists{upquote.sty}{\usepackage{upquote}}{}
\IfFileExists{microtype.sty}{% use microtype if available
  \usepackage[]{microtype}
  \UseMicrotypeSet[protrusion]{basicmath} % disable protrusion for tt fonts
}{}
\makeatletter
\@ifundefined{KOMAClassName}{% if non-KOMA class
  \IfFileExists{parskip.sty}{%
    \usepackage{parskip}
  }{% else
    \setlength{\parindent}{0pt}
    \setlength{\parskip}{6pt plus 2pt minus 1pt}}
}{% if KOMA class
  \KOMAoptions{parskip=half}}
\makeatother
\usepackage{xcolor}
\setlength{\emergencystretch}{3em} % prevent overfull lines
\setcounter{secnumdepth}{5}
% Make \paragraph and \subparagraph free-standing
\ifx\paragraph\undefined\else
  \let\oldparagraph\paragraph
  \renewcommand{\paragraph}[1]{\oldparagraph{#1}\mbox{}}
\fi
\ifx\subparagraph\undefined\else
  \let\oldsubparagraph\subparagraph
  \renewcommand{\subparagraph}[1]{\oldsubparagraph{#1}\mbox{}}
\fi

\usepackage{color}
\usepackage{fancyvrb}
\newcommand{\VerbBar}{|}
\newcommand{\VERB}{\Verb[commandchars=\\\{\}]}
\DefineVerbatimEnvironment{Highlighting}{Verbatim}{commandchars=\\\{\}}
% Add ',fontsize=\small' for more characters per line
\usepackage{framed}
\definecolor{shadecolor}{RGB}{241,243,245}
\newenvironment{Shaded}{\begin{snugshade}}{\end{snugshade}}
\newcommand{\AlertTok}[1]{\textcolor[rgb]{0.68,0.00,0.00}{#1}}
\newcommand{\AnnotationTok}[1]{\textcolor[rgb]{0.37,0.37,0.37}{#1}}
\newcommand{\AttributeTok}[1]{\textcolor[rgb]{0.40,0.45,0.13}{#1}}
\newcommand{\BaseNTok}[1]{\textcolor[rgb]{0.68,0.00,0.00}{#1}}
\newcommand{\BuiltInTok}[1]{\textcolor[rgb]{0.00,0.23,0.31}{#1}}
\newcommand{\CharTok}[1]{\textcolor[rgb]{0.13,0.47,0.30}{#1}}
\newcommand{\CommentTok}[1]{\textcolor[rgb]{0.37,0.37,0.37}{#1}}
\newcommand{\CommentVarTok}[1]{\textcolor[rgb]{0.37,0.37,0.37}{\textit{#1}}}
\newcommand{\ConstantTok}[1]{\textcolor[rgb]{0.56,0.35,0.01}{#1}}
\newcommand{\ControlFlowTok}[1]{\textcolor[rgb]{0.00,0.23,0.31}{#1}}
\newcommand{\DataTypeTok}[1]{\textcolor[rgb]{0.68,0.00,0.00}{#1}}
\newcommand{\DecValTok}[1]{\textcolor[rgb]{0.68,0.00,0.00}{#1}}
\newcommand{\DocumentationTok}[1]{\textcolor[rgb]{0.37,0.37,0.37}{\textit{#1}}}
\newcommand{\ErrorTok}[1]{\textcolor[rgb]{0.68,0.00,0.00}{#1}}
\newcommand{\ExtensionTok}[1]{\textcolor[rgb]{0.00,0.23,0.31}{#1}}
\newcommand{\FloatTok}[1]{\textcolor[rgb]{0.68,0.00,0.00}{#1}}
\newcommand{\FunctionTok}[1]{\textcolor[rgb]{0.28,0.35,0.67}{#1}}
\newcommand{\ImportTok}[1]{\textcolor[rgb]{0.00,0.46,0.62}{#1}}
\newcommand{\InformationTok}[1]{\textcolor[rgb]{0.37,0.37,0.37}{#1}}
\newcommand{\KeywordTok}[1]{\textcolor[rgb]{0.00,0.23,0.31}{#1}}
\newcommand{\NormalTok}[1]{\textcolor[rgb]{0.00,0.23,0.31}{#1}}
\newcommand{\OperatorTok}[1]{\textcolor[rgb]{0.37,0.37,0.37}{#1}}
\newcommand{\OtherTok}[1]{\textcolor[rgb]{0.00,0.23,0.31}{#1}}
\newcommand{\PreprocessorTok}[1]{\textcolor[rgb]{0.68,0.00,0.00}{#1}}
\newcommand{\RegionMarkerTok}[1]{\textcolor[rgb]{0.00,0.23,0.31}{#1}}
\newcommand{\SpecialCharTok}[1]{\textcolor[rgb]{0.37,0.37,0.37}{#1}}
\newcommand{\SpecialStringTok}[1]{\textcolor[rgb]{0.13,0.47,0.30}{#1}}
\newcommand{\StringTok}[1]{\textcolor[rgb]{0.13,0.47,0.30}{#1}}
\newcommand{\VariableTok}[1]{\textcolor[rgb]{0.07,0.07,0.07}{#1}}
\newcommand{\VerbatimStringTok}[1]{\textcolor[rgb]{0.13,0.47,0.30}{#1}}
\newcommand{\WarningTok}[1]{\textcolor[rgb]{0.37,0.37,0.37}{\textit{#1}}}

\providecommand{\tightlist}{%
  \setlength{\itemsep}{0pt}\setlength{\parskip}{0pt}}\usepackage{longtable,booktabs,array}
\usepackage{calc} % for calculating minipage widths
% Correct order of tables after \paragraph or \subparagraph
\usepackage{etoolbox}
\makeatletter
\patchcmd\longtable{\par}{\if@noskipsec\mbox{}\fi\par}{}{}
\makeatother
% Allow footnotes in longtable head/foot
\IfFileExists{footnotehyper.sty}{\usepackage{footnotehyper}}{\usepackage{footnote}}
\makesavenoteenv{longtable}
\usepackage{graphicx}
\makeatletter
\def\maxwidth{\ifdim\Gin@nat@width>\linewidth\linewidth\else\Gin@nat@width\fi}
\def\maxheight{\ifdim\Gin@nat@height>\textheight\textheight\else\Gin@nat@height\fi}
\makeatother
% Scale images if necessary, so that they will not overflow the page
% margins by default, and it is still possible to overwrite the defaults
% using explicit options in \includegraphics[width, height, ...]{}
\setkeys{Gin}{width=\maxwidth,height=\maxheight,keepaspectratio}
% Set default figure placement to htbp
\makeatletter
\def\fps@figure{htbp}
\makeatother

\usepackage{fvextra}
\DefineVerbatimEnvironment{Highlighting}{Verbatim}{breaklines,commandchars=\\\{\}}
\areaset[0.50in]{4.5in}{8in}
\makeatletter
\makeatother
\makeatletter
\@ifpackageloaded{bookmark}{}{\usepackage{bookmark}}
\makeatother
\makeatletter
\@ifpackageloaded{caption}{}{\usepackage{caption}}
\AtBeginDocument{%
\ifdefined\contentsname
  \renewcommand*\contentsname{Table of contents}
\else
  \newcommand\contentsname{Table of contents}
\fi
\ifdefined\listfigurename
  \renewcommand*\listfigurename{List of Figures}
\else
  \newcommand\listfigurename{List of Figures}
\fi
\ifdefined\listtablename
  \renewcommand*\listtablename{List of Tables}
\else
  \newcommand\listtablename{List of Tables}
\fi
\ifdefined\figurename
  \renewcommand*\figurename{Figure}
\else
  \newcommand\figurename{Figure}
\fi
\ifdefined\tablename
  \renewcommand*\tablename{Table}
\else
  \newcommand\tablename{Table}
\fi
}
\@ifpackageloaded{float}{}{\usepackage{float}}
\floatstyle{ruled}
\@ifundefined{c@chapter}{\newfloat{codelisting}{h}{lop}}{\newfloat{codelisting}{h}{lop}[chapter]}
\floatname{codelisting}{Listing}
\newcommand*\listoflistings{\listof{codelisting}{List of Listings}}
\makeatother
\makeatletter
\@ifpackageloaded{caption}{}{\usepackage{caption}}
\@ifpackageloaded{subcaption}{}{\usepackage{subcaption}}
\makeatother
\makeatletter
\@ifpackageloaded{tcolorbox}{}{\usepackage[skins,breakable]{tcolorbox}}
\makeatother
\makeatletter
\@ifundefined{shadecolor}{\definecolor{shadecolor}{rgb}{.97, .97, .97}}
\makeatother
\makeatletter
\makeatother
\makeatletter
\makeatother
\ifLuaTeX
  \usepackage{selnolig}  % disable illegal ligatures
\fi
\IfFileExists{bookmark.sty}{\usepackage{bookmark}}{\usepackage{hyperref}}
\IfFileExists{xurl.sty}{\usepackage{xurl}}{} % add URL line breaks if available
\urlstyle{same} % disable monospaced font for URLs
\hypersetup{
  pdftitle={Hands-On Pharmacoepidemiology},
  pdfauthor={Ignacio Leiva},
  hidelinks,
  pdfcreator={LaTeX via pandoc}}

\title{Hands-On Pharmacoepidemiology}
\usepackage{etoolbox}
\makeatletter
\providecommand{\subtitle}[1]{% add subtitle to \maketitle
  \apptocmd{\@title}{\par {\large #1 \par}}{}{}
}
\makeatother
\subtitle{Practical Exercises and Case Studies}
\author{Ignacio Leiva}
\date{2023-10-19}

\begin{document}
\frontmatter
\maketitle
\RecustomVerbatimEnvironment{verbatim}{Verbatim}{
   showspaces = false,
   showtabs = false,
   breaksymbolleft={},
   breaklines
   % Note: setting commandchars=\\\{\} here will cause an error 
}  

\ifdefined\Shaded\renewenvironment{Shaded}{\begin{tcolorbox}[enhanced, frame hidden, sharp corners, breakable, interior hidden, borderline west={3pt}{0pt}{shadecolor}, boxrule=0pt]}{\end{tcolorbox}}\fi

\renewcommand*\contentsname{Table of contents}
{
\setcounter{tocdepth}{2}
\tableofcontents
}
\mainmatter
\bookmarksetup{startatroot}

\hypertarget{practical-exercises}{%
\chapter*{Practical exercises}\label{practical-exercises}}
\addcontentsline{toc}{chapter}{Practical exercises}

\markboth{Practical exercises}{Practical exercises}

This hands-on workbook has been designed to help you put key PhEpi
concepts and methods into practice. This workbook supports the material
taught in the pharmacoepidemiology course and provides a practical
application framework for understanding the topics covered in lectures.
The exercises bridge the gap between theory and practice, offering
guided opportunities for deepening comprehension.

\hypertarget{lectures}{%
\section*{Lectures}\label{lectures}}
\addcontentsline{toc}{section}{Lectures}

\markright{Lectures}

\href{https://ignaleiva.github.io/Lecture2-Frequency-Effect/\#/section}{Lecture
2. Measures in (Pharmaco)Epidemiology}

\bookmarksetup{startatroot}

\hypertarget{sec-measure-of-frequency}{%
\chapter{Measures of frequency}\label{sec-measure-of-frequency}}

\hypertarget{incidence}{%
\subsection{Incidence}\label{incidence}}

\begin{Shaded}
\begin{Highlighting}[]
\DecValTok{1}\SpecialCharTok{+}\DecValTok{1}
\end{Highlighting}
\end{Shaded}

\begin{verbatim}
[1] 2
\end{verbatim}

\hypertarget{prevalence}{%
\subsection{Prevalence}\label{prevalence}}

\bookmarksetup{startatroot}

\hypertarget{sec-measure-of-effect}{%
\chapter{Measures of effect (association)}\label{sec-measure-of-effect}}

\hypertarget{risk-ratios}{%
\section{Risk Ratios}\label{risk-ratios}}

\hypertarget{excercise-1}{%
\subsection{Excercise 1}\label{excercise-1}}

You are part of a research team conducting a cohort study to evaluate
the effect of hormone replacement therapy (HRT) on the development of
coronary heart disease (CHD). The research team enrolled post-menopausal
women on HRT with no prior history of CHD and followed them for 7 years.
Please assume no loss to follow-up.

Remember: Exposure, HRT, is on the left side of the table, while the
outcome, CHD, is on the top of the table.

\begin{longtable}[]{@{}llll@{}}
\toprule\noalign{}
\endhead
\bottomrule\noalign{}
\endlastfoot
& CHD (+) & CHD (-) & Total \\
HRT (+) & 210 & 3290 & 3500 \\
HRT(-) & 250 & 6250 & 6500 \\
Total & 460 & 9540 & 10000 \\
\end{longtable}

\emph{Task:}

Calculate and interpret the relative risk (RR) for the cohort.

Results

\[
P(D+ \mid E+)= \frac{210}{3500} = 0.060
\]

\[
P(D+ \mid E-)= \frac{250}{650}0 = 0.038
\]

\[
RR= \frac{0.64}{0.4} = 1.58
\]

\textbf{Interpretation:} Women on HRT have 1.58 times the risk of CHD
compared to those who do not take HRT over 7 years.

Calculate and interpret the risk difference (RD) for the cohort

Results

\[
RD= 0.060 - 0.038 = 0.022
\]

\textbf{Interpretation:} There is an excess of 22 cases of CHD per 1000
women attributable to HRT use over 7 years.

\hypertarget{excercise-2}{%
\subsection{Excercise 2}\label{excercise-2}}

A study investigates the relationship between smoking and lung cancer. A
total of 600 people participated in the study. 250 of the 600
participants had lung cancer, and 160 of the diseased participants were
smokers. The total number of smokers in the study was 230.

Fill out the 2x2 table!

\begin{longtable}[]{@{}llll@{}}
\toprule\noalign{}
\endhead
\bottomrule\noalign{}
\endlastfoot
& Lung Cancer (+) & Lung Cancer (-) & Total \\
Smoker (+) & & & \\
Smoker(-) & & & \\
Total & & & \\
\end{longtable}

Results

\begin{longtable}[]{@{}llll@{}}
\toprule\noalign{}
\endhead
\bottomrule\noalign{}
\endlastfoot
& Lung Cancer (+) & Lung Cancer (-) & Total \\
Smoker (+) & 160 & 70 & 230 \\
Smoker(-) & 90 & 280 & 370 \\
Total & 250 & 350 & 600 \\
\end{longtable}

\emph{Task:}

What absolute part of the incidence of lung cancer is attributable to
exposure to smoking?

Results

\[
\text{Atributable Risk} = \frac{160}{230} - \frac{90}{370} =  0.69 – 0.24 = 0.45
\]

\textbf{Interpretation:} 0.45 or 45\% of the cases of lung cancer in the
cohort can be attributed to smoking

Calculate the risk ratio for the relationship between smoking and lung
cancer, interpret the result.

Results

\[
RR = \frac{\frac{160}{230}}{\frac{90}{370}} = \frac{0.69}{0.24} = 2.88
\]

\textbf{Interpretation:} The risk for smokers to develop lung cancer is
2,88 times as high compared to non-smokers.

\hypertarget{excercise-3}{%
\subsection{Excercise 3}\label{excercise-3}}

Comming back the investigation of the effect if HRT in post-menopausal
women, you now have the following information

\begin{longtable}[]{@{}lll@{}}
\toprule\noalign{}
\endhead
\bottomrule\noalign{}
\endlastfoot
& With CHD (+) & Person-Years of Disease-free Follow-up \\
HRT (+) & 30 & 54,308.7 \\
HRT (-) & 60 & 51,477.5 \\
\end{longtable}

\emph{Task:}

Calculate the Incidece rate among the exposed and unexposed women

Results

\[
IR_{exposed}= \frac{30}{54,308.7} = 0.0005524 \times 100,000 =  55.24 \text{ person-years}
\]

\[
IR_{unexposed}= \frac{60}{51,477.5} = 0.001167 \times 100,000 = 116.56 \text{ person-years}
\]

Calculate the rate ratio for the relationship between HRT and CHD,
interpret the result.

Results

\[
\text{Rate Ratio}= \frac{55.24}{116.56}  = 0.474
\]

\textbf{Interpretation:} Women on HRT had 0.47 times the rate of CHD
compared to women who did not use HRT

(Rate ratios are often interpreted as if they were risk ratios, e.g.,
post-menopausal women using HRT had 0.47 times the risk of CAD compared
to women not using HRT, but it is more precise to refer to the ratio of
rates rather than risk.)

\hypertarget{odds-ratios}{%
\section{Odds Ratios}\label{odds-ratios}}

\hypertarget{excercise-1-1}{%
\subsection{Excercise 1}\label{excercise-1-1}}

The influence of increased alcohol consumption on the incidence of
esophageal cancer was retrospectively investigated in a clinic over a
period of 5 years. Patients with other internal diseases were used as a
comparison group.

\begin{longtable}[]{@{}llll@{}}
\toprule\noalign{}
\endhead
\bottomrule\noalign{}
\endlastfoot
& Cancer (+) & Cancer (-) & Total \\
Alcohol (+) & 192 & 54 & 246 \\
Alcohol(-) & 208 & 333 & 541 \\
Total & 400 & 387 & 787 \\
\end{longtable}

\emph{Task:}

Calculate the odds to be exposed while diseased/not diseased
(i.e.~case/control)

Results

\[
Odds_{case} = \frac{192}{208} = 0.92
\]

\[
Odds_{control} = \frac{54}{333} = 0.16
\]

Calculate and interpret the odds ratio for the occurrence of esophageal
cancer

Results

\[
OR = \frac{0.92}{0.16} = 5.75
\]

\textbf{Interpretation:}

\begin{itemize}
\item
  The odds of being exposed to increased alcohol consumption among cases
  (patients with cancer) is 5.75 times as high compared to the controls
  (patients with other internal diseases).
\item
  The odds of having cancer are 5.75 times higher among individuals with
  increased alcohol consumption compared to those without increased
  alcohol consumption.
\end{itemize}

\hypertarget{excercise-2-1}{%
\subsection{Excercise 2}\label{excercise-2-1}}

A cohort study is investigating the influence of smoking on respiratory
complaints. In total, 800 people participated in the study. The
prevalence of respiratory complaints was 20\%. 300 people were smoking
and out of those, 100 participants developed respiratory complaints.

Fill out the 2x2 table!

\begin{longtable}[]{@{}llll@{}}
\toprule\noalign{}
\endhead
\bottomrule\noalign{}
\endlastfoot
& Resp. Complaints (+) & Resp. Complaints (-) & Total \\
Smoker (+) & & & \\
Smoker(-) & & & \\
Total & & & \\
\end{longtable}

Results

\begin{longtable}[]{@{}llll@{}}
\toprule\noalign{}
\endhead
\bottomrule\noalign{}
\endlastfoot
& Resp. Complaints (+) & Resp. Complaints (-) & Total \\
Smoker (+) & 100 & 200 & 300 \\
Smoker(-) & 60 & 440 & 500 \\
Total & 160 & 640 & 800 \\
\end{longtable}

\emph{Task:}

Calculate the risk and risk ratio based on the table above, interpret
the results

Results

\[
Risk_{exposed} = \frac{100}{300} = 0.33
\]

\[
Risk_{unexposed} = \frac{60}{500} = 0.12
\]

\[
RR = \frac{0.33}{0.12} = 2.75
\]

\textbf{Interpretation:} The risk of suffering from respiratory symptoms
when being a smoker is 2.75 times as high compared to participants who
didn't smoke.

Now, let's assume we designed a case-control study instead of a cohort
study with the same numbers from the 2x2 table above. Calculate the odds
and odds ratio based on the table above, interpret the result

Results

\[
Odds_{case} = \frac{100}{60} = 1.67
\]

\[
Odds_{control} = \frac{200}{440} = 0.45
\]

\[
OR = \frac{1.67}{0.45} = 3.71
\]

\textbf{Interpretation:} The odds of being exposed to smoking among
cases (patients with respiratory complaints) is 3.71 times as high
compared to the controls (patients without respiratory complaints).

\hypertarget{excercise-3-1}{%
\subsection{Excercise 3}\label{excercise-3-1}}

A second study is again investigating the influence of smoking on
respiratory complaints. This time, 900 study participants were enrolled,
93\% of those didn't suffer from respiratory complaints. 5\% of the 600
participants that were smokers suffered from respiratory complaints.

Fill out the 2x2 table!

\begin{longtable}[]{@{}llll@{}}
\toprule\noalign{}
\endhead
\bottomrule\noalign{}
\endlastfoot
& Resp. Complaints (+) & Resp. Complaints (-) & Total \\
Smoker (+) & & & \\
Smoker(-) & & & \\
Total & & & \\
\end{longtable}

Results

\begin{longtable}[]{@{}llll@{}}
\toprule\noalign{}
\endhead
\bottomrule\noalign{}
\endlastfoot
& Resp. Complaints (+) & Resp. Complaints (-) & Total \\
Smoker (+) & 30 & 570 & 600 \\
Smoker(-) & 33 & 267 & 300 \\
Total & 63 & 837 & 900 \\
\end{longtable}

\emph{Task:}

Calculate the risk ratio and the odds ratio

Results

\[
RR = \frac{\frac{30}{600}}{\frac{33}{300}} = 0.45
\]

\[
OR = \frac{\frac{30}{570}}{\frac{33}{267}} = 0.43
\]

Looking at the different measures you calculated, explain why the RR and
OR are different/similar from each other in task 2 vs.~task 3.

Results

\textbf{Rare disease assumption:} when studying diseases that have a low
prevalence (\textless10\% as a rule of thumb) in the study population,
the odds ratio is a good approximation of the risk ratio.

\bookmarksetup{startatroot}

\hypertarget{sec-bias}{%
\chapter{Bias \& Condounding}\label{sec-bias}}

\hypertarget{incidence-1}{%
\subsection{Incidence}\label{incidence-1}}

\hypertarget{prevalence-1}{%
\subsection{Prevalence}\label{prevalence-1}}


\backmatter

\end{document}
